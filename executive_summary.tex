\documentclass[a4,12pt]{article}

\usepackage[francais]{babel}
\usepackage[utf8]{inputenc}
\usepackage[T1]{fontenc}
\usepackage[babel=true]{csquotes}
\usepackage{amsmath}
\usepackage{amssymb}
\usepackage{float}
\usepackage{graphicx}
\usepackage{wrapfig}
\usepackage{hyperref}
\usepackage{array,multirow,makecell}
\usepackage[
	backend=biber,
	style=alphabetic,
	sorting=ynt
]{biblatex}

\addbibresource{biblio.bib}
\frenchbsetup{StandardLists=true}
\usepackage{enumitem}
\setlength\parindent{20pt}

\title{CECA: Ejam}
\author{Fischman Adrien, Jaskolski Aliocha, Baumann Matthieur,\\ Cornebize Tom, Thiolliere Guillaume}

\begin{document}

\maketitle

E-Jam est un logiciel pour créer et rejoindre des sessions de musique. Une session permet à ses membres de jouer ensemble, directement comme s’ils étaient dans la même salle. Les sessions créées peuvent aussi être écoutées en live via une plate-forme en ligne.

\section{Marché}

Nous nous sommes entretenus avec plusieurs contacts venant de milieux professionnels
variés afin d’en savoir plus sur l’existence d’un marché.
La plupart de nos contacts, à savoir plusieurs gérants de magasins
de musique, un ingénieur du son ainsi que des connaissances personnelles ont été
enthousiaste par rapport à l'idée notamment car cela permettrait de jouer de la musique
à plusieurs plus facilement.
\\

Nous avons aussi réalisé un sondage qui révèle que parmis les répondants
86\% sont intéréssés par jouer avec des amis.
Notons que 50\% des sondés
sont musiciens alors que le taux de persone qui pratique d'un instrument de musique en
France est de l'ordre de 10\% selon une étude du DEPS « Les pratiques culturelles des
Français à l’ère du numérique - Année 2008 »
\\

Notre questionnaire révèle aussi que 27\% des sondés seraient prêt à payer un abonnement,
et 30\% seraient prêt à participer à une campagne de financement.
D'où une réelle demande pour ce genre de service.
\\

Nous nous sommes intéressés aux sites concurrents qui proposeraient des sessions
live de musique. Même si l’offre est peu développée, quelques sites proposent ce genre de service.
Néanmoins seul Ninjam semble avoir réussi à fédérer une communauté et celui-ci propose une expérience
différente avec une mise en place de délai d'une mesure entre les musiciens.
\\

\section{Stratégie}

Notre projet est de permettre à tous les musiciens de pouvoir jouer ensemble,
depuis chez eux.
\\

Pour cela nous proposons une recherche simple de session de jeu en utilisant différents critères:
les sessions occupées par ses amis,
les sessions en accords avec son style de musique (rock, jazz, improvisation),
les sessions correspondants à son niveau de jeu.\\

Ainsi qu'une interface pour permettre de gérer les sessions: tchat, partitions, playback, equalizer etc... \\


Enfin pour développer la communauté nous ajouterons au logiciel
l’interface pour les spectateurs, nous organiserons des évènements avec des
guests et nous permettrons à des groupes de se faire connaître notamment
avec l’aide de tremplins.
\\

Ejam sera disponible en version gratuite limité en fonctionnalité et en version premium.

Ce modèle nous permettra de faire connaitre notre plateforme plus facilement et de créer une communauté importante.

Le service premium permettra un usage complet de la plateforme avec plus d'outils disponible dans les sessions notamment pour gérer l'aspet public avec les spectateurs.

\end{document}
