\documentclass[a4,12pt]{article}

\usepackage[francais]{babel}
\usepackage[utf8]{inputenc}
\usepackage[T1]{fontenc}
\usepackage[babel=true]{csquotes}
\usepackage{amsmath}
\usepackage{amssymb}
\usepackage{float}
\usepackage{graphicx}
\usepackage{wrapfig}
\usepackage{hyperref}
\usepackage{array,multirow,makecell}
\usepackage{eurosym}

\usepackage[
	backend=biber,
	style=alphabetic,
	sorting=ynt
]{biblatex}

\addbibresource{biblio.bib}
\frenchbsetup{StandardLists=true}
\usepackage{enumitem}
\setlength\parindent{20pt}

\title{CECA: Ejam}
\author{Fischman Adrien, Jaskolski Aliocha, Baumann Matthieur,\\ Cornebize Tom, Thiolliere Guillaume}

\begin{document}

\maketitle

E-Jam est un service web permettant à plusieurs musiciens de jouer ensemble en direct.
En connectant votre instrument à votre ordinateur, vous pourrez jouer à plusieurs,
chacun depuis chez soi grâce à internet.
Vous pourrez également écouter librement et en direct des personnes jouer.

\section{Marché}

Nous nous sommes entretenus avec plusieurs contacts venant de milieux professionnels
variés afin d’en savoir plus sur l’existence d’un marché.
La plupart de nos contacts, à savoir plusieurs gérants de magasins
de musique, un ingénieur du son ainsi que des connaissances personnelles ont été
enthousiaste par rapport à l'idée notamment car cela permettrait de jouer de la musique
à plusieurs plus facilement.
\\

Nous avons aussi réalisé un sondage qui révèle que parmi les répondants
86\% sont intéressés par jouer avec des amis.

Notre questionnaire révèle aussi que 27\% des sondés seraient prêts à payer un abonnement,
et 30\% seraient prêt à participer à une campagne de financement.
D'où une réelle demande pour ce genre de service.
\\

Nous nous sommes intéressés aux sites concurrents qui proposeraient des sessions
live de musique. Même si l’offre est peu développée, quelques sites proposent ce genre de service.
Néanmoins, seul Ninjam semble avoir réussi à fédérer une communauté et celui-ci propose une expérience
différente avec une mise en place de délai d'une mesure entre les musiciens.
\\

\section{Stratégie}

Notre projet est de permettre à tous les musiciens de pouvoir jouer ensemble,
depuis chez eux.
\\

Pour cela nous proposons une recherche simple de session de jeu en utilisant différents critères :
les sessions occupées par ses amis,
les sessions en accords avec son style de musique (rock, jazz, improvisation),
les sessions correspondants à son niveau de jeu.\\

Après avoir rejoint une session, l'utilisateur a à disposition une interface offrant plusieurs outils : tchat, partitions, playback, equalizer, etc.\\


Enfin pour développer la communauté nous ajouterons au logiciel
l’interface pour les spectateurs, nous organiserons des évènements avec des
musiciens renommés et nous permettrons à des groupes de se faire connaître notamment
avec l’aide de tremplins.
\\

Ejam sera décliné sous deux versions. Une version gratuite, offrant les fonctionnalités essentielles, et en version premium, apportant plus de fonctionnalités et de confort d'utilisation.

Ce modèle nous permettra de faire connaître notre plate-forme plus facilement et de créer une communauté importante.

Le service premium permettra un usage complet de la plate-forme, avec notamment un panel plus larges de partitions, de playbacks et d'effets sonores, la possibilité d'enregistrer des sessions ou de faire des sessions privées, ou bien de désactiver les publicités du site web.\\

En nous basant sur des revenus liés aux comptes premium et à la publicité, nous prévoyons de réaliser un bénéfice net de 170 000 \euro\ en 2020.


\end{document}
