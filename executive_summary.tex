\documentclass[a4,12pt]{article}

\usepackage[francais]{babel}
\usepackage[utf8]{inputenc}
\usepackage[T1]{fontenc}
\usepackage[babel=true]{csquotes}
\usepackage{amsmath}
\usepackage{amssymb}
\usepackage{float}
\usepackage{graphicx}
\usepackage{wrapfig}
\usepackage{hyperref}
\usepackage{array,multirow,makecell}
\usepackage[
	backend=biber,
	style=alphabetic,
	sorting=ynt
]{biblatex}

\addbibresource{biblio.bib}
\frenchbsetup{StandardLists=true}
\usepackage{enumitem}
\setlength\parindent{20pt}
\begin{document}

\section{Ejam}

E-Jam est un logiciel pour créer et rejoindre des sessions de musique. Une session permet à ses membres de jouer ensemble, directement comme s’ils étaient dans la même salle. Les sessions créées peuvent aussi être écoutées en live via une plate-forme en ligne.

\subsection{marché}

Nous nous sommes entretenu avec plusieurs contacts venant de milieux professionnels variés afin d’en savoir plus sur l’existence d’un marché qui nous serait profitable. De façon globale, l’ensemble de nos contacts, à savoir plusieurs gérants de magasins de musique, un ingénieur du son ainsi que des connaissances personnelles ont été conquis par l’idée. Ainsi, nombreux sont ceux qui sont intéressés par une telle plateforme Web interactive leur permettant de jouer avec leur amis et en temps réel. Néanmoins certains ont émis quelques réticences quand à la faisabilité d'un tel projet en raison de problèmes liés à la latence.
\\


Nous avons aussi réalisé un sondage, dans ce sondage nous notons que 50\% des répondants sont musiciens alors que le taux de persone qui pratique d'un instrument de musique en France est de l'ordre de 10\% selon une étude du DEPS « Les pratiques culturelles des Français à l’ère du numérique - Année 2008 »

Parmi l’intégralité des répondants, 82\% sont intéressés par écouter des artistes renommés, 75\% des amis, et 57\% des inconnus. Cela montre que les répondants semblent moins intéressés par de la musique jouée par des inconnus que par leurs amis. De plus, ils semblent plus intéressés par écouter des artistes renommés que par jouer avec eux.

Notre questionnaire révèle que parmi les répondants, 27\% seraient prêt à payer un abonnement, et 30\% seraient prêt à participer à une campagne de financement. Ces nombres sont assez satisfaisants, 27\% d’utilisateurs abonnés nous semble être une bonne proportion.
\\


Nous nous sommes intéressés à l’offre déjà disponible sur internet, proposant des sessions live de musique. Même si l’offre est peu développée, une poignée de sites proposant des services similaires existent déjà.

Il y a notamment Ninjam un logiciel libre qui dispose d'une communauté relativement active en France.
Ainsi que Sofasession et Jammr mais qui n'ont qu'une communauté très réduite

\subsection{stratégie}

Notre projet profite des avancés technologiques en matière de réseaux : amélioration des débits, diminution du temps de latence et robustesse de la connexion. Cela permet de nous différencier des concurrents qui ont choisi de faire jouer les musiciens avec un délai (une mesure pour Ninjam), technique utilisée afin de palier les problèmes de latence que posait le réseau internet de l’époque.
\\

D’autre part nous proposons une recherche simple de session de jeu en utilisant différent critères:
\begin{itemize}
	\item les sessions occupées par ses amis
	\item les sessions en accord avec son style de musique (rock, jazz, improvisation)
	\item les sessions correspondant à son niveau de jeu.
\end{itemize}

Enfin pour développer la communauté nous ajouterons au logiciel
l’interface pour les spectateurs, nous organiserons des évènements avec des
guests et nous permettrons à des groupes de se faire connaître notamment
avec l’aide de tremplins.
\\

Ejam sera disponible en version gratuite limité en fonctionnalité et en version premium.

Ce modèle nous permettra de faire connaitre notre plateforme plus facilement et de créer une communauté importante.

Le service premium permettra un usage complet de la plateforme avec plus d'outils disponible dans les sessions notamment pour gérer l'aspet public avec les spectateurs.

\end{document}
